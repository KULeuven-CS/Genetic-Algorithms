\documentclass[../main.tex]{subfiles}

\begin{document}

\begin{mdframed}
\fullcite{6036171}
\end{mdframed}

\begin{abstract}
This paper presents ACROMUSE, a novel genetic algorithm (GA) which adapts crossover, mutation, and selection
parameters. ACROMUSEs objective is to create and maintain a diverse population of highly-fit (healthy) individuals,
capable of adapting quickly to fitness landscape change and well-suited to the efficient optimization of multimodal
fitness landscapes. A new methodology is introduced for determining standard population diversity (SPD) and an original
measure of healthy population diversity (HPD) is proposed. The SPD measure is employed to adapt crossover and mutation,
while selection pressure is controlled by adapting tournament size according to HPD. In addition to selection pressure
control, ACROMUSE tournament selection selects individuals according to healthy diversity contribution rather than
fitness. This proposed selection mechanism simultaneously promotes diversity and fitness within the population. The
performance of ACROMUSE is evaluated using various multimodal benchmark functions. Statistically significant results
are presented comparing ACROMUSEs fitness and diversity performance to that of several other GAs. By maintaining a
diverse population of healthy individuals, ACROMUSE responds to fitness landscape change by restoring better fitness
scores faster than other GAs. Analysis of the adaptive operators illustrates that the key benefit of ACROMUSE is the
synergy of the operators working together to achieve an effective balance between exploration and exploitation.
\end{abstract}

\end{document}