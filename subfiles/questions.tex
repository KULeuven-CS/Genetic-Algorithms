\documentclass[../main.tex]{subfiles}

\begin{document}

\begin{enumerate}
    \question{Explain Holland's schema theorem. Which assumptions that are made in this theorem and are somewhat
    unlikely?}{ The schema theorem considers the canonical GA with binary string representation of individuals with an
    alphabet $\{0,1,\#\}$ where the latter is a wild card symbol. The theorem says that the number of short schemata
    with low order and above average quality, grows exponentially in subsequent generations of a genetic algorithm.
    This result is obtained by considering the influence of replication, crossover and mutation. It is unlikely because
    it is obtained under idealized conditions that do not hold for most practical GA applications. Furthermore, very
    few problems can be modeled as the canonical GA (binary encoding, proportional selection, single-point crossover,
    strong mutation). }
    \question{Explain, chapter 4, page 69, the sentence ``Both algorithms are self-adaptive \textellipsis'' (About
    Offspring Selection and RAPGA)}{This means that both algorithms alter their selection based on the state of the
    current population and their offspring. Offspring selection does this by requiring a certain percentage of children
    to outperform their parents. RAPGA on the other hand increases and decreases the population size relative to the
    amount of children that both outperform their parents and contain new genetic information.}
    \question{Suppose you want to solve an optimization problem, for which the objective function (function to be
    maximized) is known. Why it is usually not appropriate to use the objective function as the fitness function? Give
    several reasons if possible. How can you construct a good fitness function? Give an example of such a construction
    of a fitness function in one of the texts that we have discussed.}{\unanswered}
    \question{Given the charts in the article on GABIL, can you say that the tree based concept learner (ID5R)
    outperforms the genetic algorithm? Discuss figures 1 -- 7 in detail.}{\unanswered}
    \question{Explain the terms ``epistasis'' and ``deception''. What is the difference?}{
    \textbf{Epistatis} indicates that there is a nonlinear interaction among the bits of the string. This means that
    the effect of one gene is being dependent on the presence of another. A search function is \textbf{deceptive} when
    the low-order high-fitness value schemata do not contain the optimal string as an instance. An example is the
    minimal deceptive problem, where $f(11) > f(\#\#)$ but $f(\#0) > f(\#1)$ or $f(0\#) > f(1\#)$.}
    \question{Discussion of rank-based selection. Which parameters can be used to adapt the behavior of these methods?
    When would you want to use a rank-based selection method?}{ In the context of linear-rank selection the individuals
    of the population are ordered according to their fitness and copies are assigned in such a way that the best
    individual receives a pre-determined multiple of the number of copies the worst one receives. On the one hand rank
    selection implicitly reduces the dominating effects of ``super individuals'' in populations (i.e., individuals that
    are assigned a significantly better fitness value than all other individuals), but on the other hand it warps the
    difference between close fitness values, thus increasing the selection pressure in stagnant populations. Even if
    linear-rank selection has been used with some success, it ignores the information about fitness differences of
    different individuals and violates the schema theorem}
    \question{Discussion of the paper Evolving 3D morphology and behavior by competition by K. Sims. The fitness
    function is different from the fitness functions we are used to. Explain how and why.}{\unanswered}
    \question{Handling constraints: Explain the difference between algorithms based on repair methods and algorithms
    based on decoders. Why these strategies can lead to a better performance compared with algorithms based on penalty
    functions? How would you take into account the constraints arising in a timetabling problem?}{\unanswered}
    \question{What techniques are used in order to obtain better load balance? Can these be used in other
    applications?}{\unanswered}
    \question{Discuss some drawbacks of the roulette wheel selection mechanism, and describe some methods to avoid (or
    to minimize) these drawbacks.}{ Fitness proportionate select can cause premature convergence because outstanding
    individuals quickly take over the entire population. The dominance of a single group of highly fit individuals
    (``super individuals'') can be reduced by stochastic sampling techniques. The second drawback is caused when the
    fitness values are close together,leading to low selection pressure. This can be mitigated by so called ``windowing
    techniques'' to make the selection independent of the dimension of the fitness value. }
    \question{What is evolutionary programming? Can this still be considered a genetic algorithm? Give one ore two
    examples to illustrate the approach.}{ Genetic programming is an extension of the GA, where the population is a
    computer programming. The same principles of selection, crossover and mutation are applied to come to a solution.
    The most common approach represents programs as structured syntax trees. A notable difference between GP and GA is
    that in GP, crossover and mutation (or a simple copy) are executed independently. Each time new offspring is to be
    created, one of these variants is chosen probabilistically, as opposed to GA, where they are applied sequentially.
    }
    \question{Consider varying population sizes for GA's. Discuss (1) the purpose, (2) the importance of the life
    parameter and (3) the relevant figure in the book.}{ Assuming this questions regards RAPGA and Figure~4.3 from the
    book \cite{affenzeller2009genetic} and that the life parameter is the minimum and maximum population size. The
    purpose of the varying population size is maintaining as much genetic diversity while making as much progress as
    possible from the actual population. The population size requires a upper limit because the population would
    otherwise explode in the first rounds which would be very inefficient. The lower limit is intended to maintain a
    sufficient amount of chromosomes to outperform their parents. Reaching the lower bound can be used as an indicator
    for convergence. Figure~3.4 shows how the population size changes between the lower and upper bound.}
\end{enumerate}

\end{document}